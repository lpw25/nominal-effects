\documentclass{article}
\usepackage{supertabular}
\usepackage{bm}
\usepackage{mathpartir}
\usepackage{amsmath,amssymb}
\usepackage{stmaryrd}
\usepackage{supertabular}
\usepackage{geometry}
\usepackage{ifthen}
\usepackage{alltt}%hack
\geometry{a4paper,dvips,twoside,left=22.5mm,right=22.5mm,top=20mm,bottom=30mm}
\usepackage{color}

\usepackage{listings}
\lstdefinestyle{ocaml}
{
  language=[Objective]Caml,
  escapeinside={(**}{*)},
  morekeywords={handle, return, effect, shift},
  basicstyle=\ttfamily,
  commentstyle=\color{green},
  keywordstyle=\color{blue},
  stringstyle=\color{mauve},
}

\lstset{
  showstringspaces=false,
  style=ocaml,
}
\newcommand{\code}[2][ocaml]{\lstinline[style={#1}]{#2}}

\input{nominal-effects-gen.tex}

\renewcommand{\ottusedrule}[1]{#1 \and}
\renewenvironment{ottdefnblock}[3][]
  { \framebox{ \mbox{#2} } \quad \begin{mathpar}#3 }
  { \end{mathpar} }

\title{Nominal algebraic effects}
\author{Leo White}
\date{}
\begin{document}
\maketitle
\section{Introduction}
Algebraic effects were originally introduced\cite{??} to study the
semantics of computational effects. With the addition of
handlers\cite{??} they have become an exciting new programming construct
for implementing such effects. They have appeared in a number of
experimental language\cite{??} allowing convenient implementations of
other important language features, including exceptions, coroutines,
generators, state, and cooperative concurrency.
Whilst existing implementations of algebraic effects have shown them to
be very useful, there are some subtle issues around the design of
algebraic effect systems that a number of recent papers have tried to
address\cite{??}. In this paper, we suggest that these issues be framed
as instances of the age-old problem of how to represent structures with
binding, and propose using the powerful tools from nominal
techniques\cite{??} to address them.

\subsection{Relating operations and handlers}

The core idea behind algebraic effects is to separate purely syntactic
effectful operations from their semantic interpretation via
handlers. For example, a computation might use the \code{Get} and
\code{Set} operations of a state effect:
\begin{lstlisting}
let incr () =
  let prev = Get () in
  Set (prev + 1)
\end{lstlisting}
These operations can then be given a meaning by placing the compuation
within an effect handler:
\begin{lstlisting}
let three =
  let run =
    handle incr (); incr (); incr () with
    | return x -> (fun s -> s)
    | Get (), k -> (fun s -> k s s)
    | Set y, k -> (fun _ -> k () y)
  in
  run 0
\end{lstlisting}
The effect handler defines how each operation of the effect is
interpreted along with a \code{return} clause giving an interpretation
for pure values.

Separating the operations and the handlers like this makes it easy to
define computations which use multiple effects by just using operations
from different effects within a single expression. To handle such
expressions we simply enclose them in multiple effect handlers. These
effect handlers, and the order in which they are placed, define
the semantics of the combined effects.

For example, this computation uses the \code{Raise} operation from an
exception effect in combination with a state effect:
\begin{lstlisting}
let least less_than list =
  match list with
  | [] -> Raise ()
  | first :: rest ->
      Set first;
      List.iter
        (fun item -> if less_than next (Get ()) then Set next)
        rest;
      Get ()
\end{lstlisting}
and it can be given a meaning using a two handlers:
\begin{lstlisting}
let handle_state init f =
  let run =
    handle f () with
    | return x -> (fun _ -> x)
    | Get (), k -> (fun s -> k s s)
    | Set y, k -> (fun _ -> k () y)
  in
  run init

let least_int list =
  let less_than x y = x < y in
  handle
    handle_state 0
     (fun () -> least less_than list)
  with
  | return x -> x
  | Raise (), k -> 0
\end{lstlisting}

In cases like this where there is a single handler for each kind of
effect it is obvious which handler handles each operation. It is simply
a matter of tracking which handlers enclose each operation and then
relating the operation to the enclosing handler that handles that
operation.

However, things become more difficult when there are multiple handlers
for the same effect. Consider the following code, which uses the
\code{handle_sate} function from the previous example:
\begin{lstlisting}
let map f l =
  List.rev
    (handle_state []
      (fun () -> List.iter (fun i -> Set (f i) :: Get()) l))

let multiply_by_index l =
  handle_state 0
    (fun () -> map (fun i -> Set (Get() + 1); i * Get()))
\end{lstlisting}
Here we have two nested handlers for the state effect. It is clear the
intention is for the uses of \code{Set} and \code{Get} in the \code{map}
function to be handled by the state handler introduced in the \code{map}
function, whist the uses of \code{Set} and \code{Get} in the
\code{multiply_by_index} function are intended to be handled by the
state handler introduced in the \code{multiply_by_index}
function. However, the nearest enclosing state handler for all the
operations is the one introduced by the \code{map} function, and so by
default that is the one that existing systems would choose to handle
all of the operations.

The original semantic presentation of algebraic effects considered the
operations as families of algebraic operations on computations. From
that perspective, the an application of \code{multiply_by_index} looks
something like:
\begin{lstlisting}
  handle
    handle
      Get()(fun acc ->
        Get()(fun idx ->
          Set(idx + 1)(fun () ->
            Get()(fun idx ->
              Set(idx * i :: acc)(fun () ->
                ...
              )
            )
          )
        )
      )
    with
    | return x -> (fun _ -> x)
    | Get (), k -> (fun s -> k s s)
    | Set y, k -> (fun _ -> k () y)
  with
  | return x -> (fun _ -> x)
  | Get (), k -> (fun s -> k s s)
  | Set y, k -> (fun _ -> k () y)
\end{lstlisting}
Presented this way, it is clear that handlers essentially bind the names
of operations, and that the issue with \code{multiply_by_index} is one
of accidental capture of bound names.

Existing algebraic effect systems solve this issue, if they solve it at
all, by provide a \code{shift} or \code{lift} construct that allows the
nearest enclosing effect handler to be hidden from a
sub-expression. This allows us to fix \code{multiply_by_index} by
changing the definition of \code{map} to:
\begin{lstlisting}
let map f l =
  List.rev
    (handle_state []
      (fun () -> List.iter (fun i -> Set (shift[Get, Set] (f i)) :: Get()) l))
\end{lstlisting}
This approach is equivalent to using Debruijn indices: operations have
an index that determines how many enclosing handlers must be passed
through before it can be handled, and \code{shift} increments the
indices of all operations it encloses. Debruijn indices are a popular
method for representing binding in data structures and algorithms, but
they are generally considered to be too cumbersome for people to use
directly.

A similar problem arises when trying to abstract over effects. ?? et
al.\cite{??} describe, as a motivating example, creating an interface
for union-find:
\begin{lstlisting}
type 'a set
effect 'a uf
val new : 'a -['a uf]-> 'a set
val find : 'a set -['a uf]-> 'a
val union : ('a -> 'a -['r]-> 'a) -> 'a set -> 'a set -['a uf | 'r]-> unit
val with_uf : (unit -['a uf | 'r]-> 'b) -['r]-> 'b
\end{lstlisting}
where the union-find operations \code{new}, \code{find} and \code{union}
all perform some effect \code{uf}, which can be handled using the
\code{with_uf} function. By keeping \code{uf} abstract, the interface
ensures that all the operations performed by \code{new}, \code{find} and
\code{union} will only ever be handled by \code{with_uf} and thus they
can rely on invariants guaranteed by that handler.

However, consider code which mixes this abstract effect with other
effects:
\begin{lstlisting}
handle_state 0 (fun () ->
  with_uf (fun () ->
    ...; Get (); ...
  )
)
\end{lstlisting}
Whether or not this code works as intended depends on whether or not the
\code{uf} effect is defined as the state effect or not. If \code{uf}
is defined as the state effect then we should instead use:
\begin{lstlisting}
handle_state 0 (fun () ->
  with_uf (fun () ->
    ...; shift uf Get (); ...
  )
)
\end{lstlisting}
but if \code{uf} is not defined as the state effect then the first
version was right and this second version is wrong. ?? et al. address
this issue by provide more coercions similar to \code{shift} that allow
the above code to be written in a way that will work regardless of the
effect choosen for \code{uf}. However, this has reduced us to using the
equivalent of Debruijn indices for all our effects as soon as one of
them is made abstract.

This issue comes from the fact that existing algebraic effect systems
use a single notion of effect type to represent both which operations an
effect contains -- which we would like to make abstract -- and how
operations should be related to handlers -- which we would like to keep
concrete.

\subsection{Nominal techniques}

Nominal techniques\cite{??} are an approach to handling structures with
binding based around \emph{permutations} of names and the notion of
\emph{freshness}. They have been used in a wide variety of
applications\cite{??} to handle terms with binding.

A number of languages have been proposed with built-in support for
nominal operations\cite{??}. Such languages support generating fresh
names, name abstraction, fresh name abstraction, and application of
permutations. These operations allow manipulation of structures with
binding, and aim to be more convenient to use than other techniques,
such as Debruijm indices or Higher-order Abstract Syntax.

\subsection{Contributions}

\begin{itemize}
\item We introduce nominal algebraic effects as a convenient approach to
  managing the relationship between operations and handlers.
\item We provide a core calculus implementing nominal algebraic
  effect. We define the type-and-effect system and operational semantics
  of this calculus, and show they are sound using logical relations.
\item We extend the core calculus to support \emph{shift-permutations}
  to enable Debruijn-style handling of effects in a nominal setting.
\item We extend the core calculus to support recursion. This involves
  adding support for recursive types to accurately describe the
  principal types of recursive expressions.
\item We provide an implementation of a simple surface language based on
  the core calculus.
\end{itemize}

\section{Managing effects with nominal techniques}

% Give an overview of how nominal effects works
% Show how the examples from the introduction are handled
% Show some additional example of why the system is good
\section{Core calculus}
\subsection{Grammer}
\subsubsection{Meta-variables}
\ottmetavars
\subsubsection{Productions}
\ottgrammar
\subsection{Typing judgements}
\ottdefnsKinding
\ottdefnsEquality
\ottdefnsSubtyping
\ottdefnsTyping
\section{Soundness}
% Demonstrate soundness of the system
% Preferably using logical relations
% but if that's too hard just do preservation and progress
\section{Extending the core calculus with shift}
% Show an example where shift is useful
% Describe the changes to the core calculus needed to support it
\section{Extending the core calculus with recursion}
% Show an example using recursion whose principal type requires recursive types
% Describe the changes to the core calculus needed to support it
\section{Prototype implementation}
% Short description of prototype implementation
% Main thing is to give an outline of how/why type inference works
\section{Related work}
\section{Future work}
\end{document}